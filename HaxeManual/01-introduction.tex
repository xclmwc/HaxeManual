\chapter{简介}
\label{introduction}
\state{NoContent}

\section{Haxe有什么特点?}
\label{introduction-what-is-haxe}

Haxe是一门开源的高级语言,有自己的编译器,符合ECMScript语法规范\footnote{http://www.ecma-international.org/publications/standards/Ecma-327.htm}。并且,Haxe是跨平台的,可以编译为多种目标平台的原生语言,这样就可以保持只要一份代码,就能编译到多个平台。

Haxe是强类型的,但如果需要也可以改变这种特性。因为有类型信息,Haxe可以在编译期发现错误,即使该错误是目标语言在运行期才会发现。并且,目标生成器可以利用类型信息生成优化的健壮代码。

目前可以编译为以下9中目标语言:

\begin{center}
\begin{tabular}{| l | l | l |}
	\hline
	语言 & 输出类型 & 平台 \\ \hline
	JavaScript & 源代码 & 浏览器, 桌面, 移动平台, 服务器 \\
	Neko & 字节码 & 桌面, 服务器 \\
	PHP & 源代码 & 服务器 \\
	Python & 源代码 & 桌面, 服务器 \\
	C++ & 源代码 & 桌面, 移动平台, 服务器 \\
	ActionScript 3 & 源代码 & 浏览器, 桌面, 移动平台 \\
	Flash & 字节码 & 浏览器, 桌面, 移动平台 \\ 
	Java & 源代码 & 桌面, 服务器 \\
	C\# & 源代码 & 桌面, 移动平台, 服务器 \\ \hline
\end{tabular}
\end{center}

本章剩余部分是Haxe的简介和自2005年诞生以来的发展历史。

\todo{interact,unification该怎么翻译}
\Fullref{types} 介绍Haxe的7中类型(and后觉得可以不翻译)。像\emph{结构体},\emph{类型参数},\emph{类型推断}等特性将在\Fullref{type-system}讲解。

\Fullref{class-field}讲解Haxe中的类。以及如何使用\emph{属性},\emph{内联域},\emph{泛型函数}。

\Fullref{expression}则会讲解如何使用表达式。

\Fullref{lf}是Haxe语法部分的最后一章。讲解了一些Haxe的特性,如\emph{模式匹配},\emph{字符串插值},\emph{消除冗余代码}。

接下来的部分是关于Haxe编译器的。\Fullref{compiler-usage}是关于编译器的基础用法,而\Fullref{cr-features}则介绍了更多的编译特性。最后会在\Fullref{macro}中详细讲解\emph{haxe macros}。Macro是一个非常强大的语言特性,能简化许多相同的任务处理。

在接下来的章节中,\Fullref{std}将探索Haxe标准库中的关键数据类型及其操作方法;而\Fullref{haxelib}则会讲解Haxe库管理器Haxelib。

Haxe统一了许多平台差异,但是有时候还是需要对特定平台做处理,关于其详细信息会在\Fullref{target-details}里介绍。

\section{关于本书}
\label{introduction-about-this-document}

\todo{ and there are references to topics ``previously seen'' and topics ``yet to come''该怎么翻译}
本书是Haxe 3的官方手册,所以不太适合编程初学者,不过各个章节是循序渐进的。在能简化讲解的时候可能会用到后面章节的知识,不过不用担心会影响阅读。

我们用了大量的Haxe源代码来说明如何实际使用理论知识。一般来说,那些源代码都是完整的程序,可以直接编译测试,不过有时候仅仅是部分重要代码。
源代码样式:

\begin{lstlisting}
Haxe源代码
\end{lstlisting}
偶尔,我们会用\target{JavaScript}源代码来展示Haxe是如何编译的。

此外,主要在介绍一个新类型或者对于Haxe来说是新术语时,会在书中定义该术语,不过不会定义所有术语。例如:一个类的具体定义。
定义样式:

\define{定义名}{定义释义}{定义解释}

书中有些地方会有\emph{冷知识}。用于说明一些不太重要的信息。如Haxe发展中为什么做出这样的决定,Haxe过去版本的特性变化。你可以跳过这些知识。
冷知识样式:

\trivia{简介}{说明}

\subsection{作者和他们的贡献}
\label{introduction-authors-and-contributions}

本书大部分内容由Simon Krajewski编写,他也是Haxe Foundation的成员之一。感谢那些做出贡献的人。

\begin{itemize}
	\item Dan Korostelev: 额外内容和校订
	\item Caleb Harper: 额外内容和校订
	\item Josefiene Pertosa: 校订
	\item Miha Lunar: 校订
	\item Nicolas Cannasse: Haxe之父
\end{itemize}

\subsection{版权}
\label{introduction-license}

由\href{http://haxe.org/foundation}{Haxe Foundation}编写的本书符合\href{http://creativecommons.org/licenses/by/4.0/}{Creative Commons Attribution 4.0 International License}许可协议。

该书内容保存在\href{https://github.com/HaxeFoundation/HaxeManual}{https://github.com/HaxeFoundation/HaxeManual}

\section{Hello World}
\label{introduction-hello-world}

以下代码会显示``Hello World'':

\haxe{assets/HelloWorld.hx}
将代码保存在\ic{Main.hx}中,然后在命令行运行:\ic{haxe -main Main --interp}。输出:\ic{Main.hx:3: Hello world}。我们可以从中了解到:


\begin{itemize}
	\item Haxe源文件后缀名为\ic{.hx}.
	\item Haxe编译器是一个命令行工具,命令格式为\ic{-main Main},\ic{--interp}.
	\item Haxe有类(\type{Main},大写字母开头),有方法(\expr{main}, 小写字母开始)。
	\item 源文件名和类名相同 (in this case \type{Main.hx}). 
\end {itemize}
 
\     section{历史 }
\label{introduction-haxe-history}
\state{Reviewed}

\todo{the in-house \emph{MTypes} language怎么翻译}
Nicolas Cannasse\footnote{法国人,游戏《Evoland(进化之地)》原作者,译者注}是一名ActionScript 2编译器(\emph{MTASC},Motion-Twin Action Script Compiler)和\emph{MTypes}(在面对对象中尝试类型推断的语言)的爱好者,并在2005年10月22日开始开发Haxe。Nicolas长期以来对设计编程语言的热爱,以及\emph{Motion-Twin}游戏开发者们不同技术融合所带来的新机遇下,诞生了一门新的语言。

一开始Haxe是叫haXe,测试版本在2006年2月发布,可以编译为AVM\footnote{Adobe Virtual Machine}和Nicolas自己的虚拟机Neko\footnote{http://nekovm.org}字节码。

Nicolas Cannasse一直作为Haxe项目的主导者直到现在,并在2006年5月发布了Haxe 1.0。作为第一个主要版本,提供了对\target{JavaScript}的支持,以及包含了部分Haxe的现在特性,如类型推断,structural sub-typing。

Haxe 1艰难地发展两年,发布了多个次要版本,在2006年8月添加\target{Flash AVM2}的支持,以及haxelib工具,在2007年3月添加\target{ActionScript 3}的支持。在那段时间中,一直都在奋力提升Haxe的稳定性,所以才有了多个修复bug的版本发布。

2008年7月发布了Haxe 2.0,由\emph{Franco Ponticelli}提供了\target{PHP}的支持。2009年发布支持\target{C++}的Haxe 2.04,由\emph{Hugh Sanderson}主导开发。

就像Haxe 1一样,在数月间发布了多个稳定版本。在2011年1月,Haxe 2.07发布,增加\emph{macros}特性。\emph{Bruno Garcia}在那时候加入我们作为\target{JavaScript}的维护者,随后发布了重要改进的2.08和2.09。

在2.09发布后,\emph{Simon Krajewski}也加入项目组并开始了Haxe 3的工作。此外\emph{Cau\^{e} Waneck}为Haxe添加了\target{Java}和\target{C\#}的支持。最后在2012年7月发布Haxe 2最终版本Haxe 2.1.0。

在2012年末,Haxe编译器项目组(由现在重新建立的\emph{Haxe Foundation}\footnote{http://haxe-foundation.org}支持)开始重点开发Haxe 3。随后在2013年5月发布。
